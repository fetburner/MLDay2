\documentclass[dvipdfmx,cjk,xcolor=dvipsnames,envcountsect,notheorems,12pt]{beamer}
% * 16:9 のスライドを作るときは、aspectratio=169 を documentclass のオプションに追加する
% * 印刷用の配布資料を作るときは handout を documentclass のオプションに追加する
%   (overlay が全て一つのスライドに出力される)

\usepackage{pxjahyper}% しおりの文字化け対策 (なくても良い)
\usepackage{amsmath,amssymb,amsfonts,amsthm,ascmac,cases,bm,pifont}
\usepackage{graphicx}
\usepackage[thicklines]{cancel}
\usepackage{url}
\usepackage{newtxmath}
\usepackage{bussproofs}
\usepackage{etex}
\usepackage[all]{xy}
\usepackage{mathtools}
\usepackage[absolute, overlay]{textpos}

% スライドのテーマ
\usetheme{sumiilab}
% ベースになる色を指定できる
%\usecolortheme[named=Magenta]{structure}
% 数式の文字が細くて見難い時は serif の代わりに bold にしましょう
%\mathversion{bold}

%% ===============================================
%% スライドの表紙および PDF に表示される情報
%% ===============================================

%% 発表会の名前とか(省略可)
%% スライドのタイトル
\title{OCamlによる無限グラフにも使える\\ダイクストラ法の実装}
\subtitle{ML Day \#2}

%% 必要ならば、サブタイトルも
%% 発表者のお名前
\author{@fetburner}
%% 発表者の所属([] 内は短い名前)
%\institute[東北大学 住井・松田研]{東北大学 工学部 電気情報物理工学科\\住井・松田研究室}% 学部生
%\institute[東北大学 住井・松田研]{東北大学 大学院情報科学研究科\\住井・松田研究室}% 院生
%% 発表する日
\date{2018年9月16日}

%% ===============================================
%% 自動挿入される目次ページの設定(削除しても可)
%% ===============================================

%% section の先頭に自動挿入される目次ページ(削除すると、表示されなくなる)
%\AtBeginSection[]{
%\begin{frame}
%  \frametitle{アウトライン}
%  \tableofcontents[sectionstyle=show/shaded,subsectionstyle=show/show/hide]
%\end{frame}}
%% subsection の先頭に自動挿入される目次ページ(削除すると、表示されなくなる)
%\AtBeginSubsection[]{
%\begin{frame}
%  \frametitle{アウトライン}
%	\tableofcontents[sectionstyle=show/shaded,subsectionstyle=show/shaded/hide]
%\end{frame}}

%% 現在の section 以外を非表示にする場合は以下のようにする

%% \AtBeginSection[]{
%% \begin{frame}
%%   \frametitle{アウトライン}
%%   \tableofcontents[sectionstyle=show/hide,subsectionstyle=show/show/hide]
%% \end{frame}}
%% \AtBeginSubsection[]{
%% \begin{frame}
%%   \frametitle{アウトライン}
%%   \tableofcontents[sectionstyle=show/hide,subsectionstyle=show/shaded/hide]
%% \end{frame}}

%% ===============================================
%% 定理環境の設定
%% ===============================================

\theoremstyle{definition}
\newtheorem{definition}{定義}
\newtheorem{axiom}{公理}
\newtheorem{theorem}{定理}
\newtheorem{lemma}{補題}
\newtheorem{corollary}{系}
\newtheorem{proposition}{命題}

%% ===============================================
%% ソースコードの設定
%% ===============================================

\usepackage{listings,jlisting}
%\usepackage[scale=0.9]{DejaVuSansMono}

\definecolor{DarkGreen}{rgb}{0,0.5,0}
% プログラミング言語と表示するフォント等の設定
\lstset{
  language={[Objective]Caml},% プログラミング言語
  basicstyle={\ttfamily\small},% ソースコードのテキストのスタイル
	keywordstyle={\bf},% 予約語等のキーワードのスタイル
  commentstyle={},% コメントのスタイル
  stringstyle={},% 文字列のスタイル
  frame=trlb,% ソースコードの枠線の設定 (none だと非表示)
  numbers=none,% 行番号の表示 (left だと左に表示)
  numberstyle={},% 行番号のスタイル
  xleftmargin=5pt,% 左余白
  xrightmargin=5pt,% 右余白
  keepspaces=true,% 空白を表示する
  mathescape,% $ で囲った部分を数式として表示する ($ がソースコード中で使えなくなるので注意)
  % 手動強調表示の設定
  moredelim=[is][\itshape]{@/}{/@},
  moredelim=[is][\color{red}]{@r\{}{\}@},
  moredelim=[is][\color{blue}]{@b\{}{\}@},
  moredelim=[is][\color{DarkGreen}]{@g\{}{\}@},
}

%% ===============================================
%% 本文
%% ===============================================
\begin{document}
\frame[plain]{\titlepage}% タイトルページ

\section*{アウトライン}

\begin{frame}
	\frametitle{発表内容}
	\Large
	\begin{center}
		競プロで使うのでダイクストラ法を実装
	\end{center}
	\vfill
	実用上の要請:
	\begin{itemize}
		\item 疎なグラフに対して高速に
		\item ある始点からの最短距離を一度に計算
	\end{itemize}
	\vfill
	趣味:
	\begin{itemize}
		\item 無限グラフにも使えるようにしてみる
	\end{itemize}
\end{frame}

\begin{frame}
  \frametitle{アウトライン}
	\begin{center}
		\begin{minipage}{.9\textwidth}
			\tableofcontents[sectionstyle=show,subsectionstyle=hide]
		\end{minipage}
	\end{center}
\end{frame}

\section{ダイクストラ法とは}

\begin{frame}
	\frametitle{ダイクストラ法}
	\Large
	\begin{center}
		非負の重みが付いた有向グラフに対して,\\単一始点最短経路問題を解くアルゴリズム
	\end{center}
	\vfill
	\begin{minipage}{\linewidth}
		\begin{block}{}
			\begin{enumerate}
				\item $V$を頂点全ての集合,始点への最短距離を0,それ以外への最短距離を無限大とする
				\item $V$が空集合ならば手続きを終了 \label{enum:dijkstra-while}
				\item 最短距離が最小となる$v\in V$を$V$から削除
				\item $v$から伸びる辺で最短距離を更新
				\item \ref{enum:dijkstra-while}へ
			\end{enumerate}
		\end{block}
	\end{minipage}
\end{frame}

\section{雑実装}

\begin{frame}[fragile]
	\frametitle{線形探索による実装(1/2)}
	\Large
	\begin{lstlisting}[language=Caml]
let rec dijkstra_aux : int list ->
  (int -> (int * float) list) ->
  float array -> unit = fun q e d ->
  match q with
  | [] -> ()
  | v :: q ->
      let v, q =
        List.fold_left (fun (u, q) v ->
          if d.(u) < d.(v)
          then (u, v :: q)
          else (v, u :: q)) (v, []) q in
      List.iter (fun (u, c) ->
        d.(u) <- min d.(u) (c +. d.(v))) (e v);
      dijkstra_aux q e d
\end{lstlisting}
\end{frame}

\begin{frame}[fragile]
	\frametitle{線形探索による実装(2/2)}
	\Large
	\begin{lstlisting}[language=Caml]
let dijkstra : int ->
  (int -> (int * float) list) ->
  int -> int -> float = fun n e s ->
  let d = Array.make n infinity in
  d.(s) <- 0.;
  dijkstra_aux (List.init n (fun v -> v)) e d;
  fun v -> d.(v)
\end{lstlisting}
	\begin{itemize}
		\item 実装を隠蔽するため,関数を返す
		\item 最も近い頂点を線形探索しているため,\\
			疎なグラフに対しては遅い($O(V^2)$)
	\end{itemize}
\end{frame}

\section{Mapをヒープとして使う}

\begin{frame}[fragile]
	\frametitle{Mapをヒープとして使う(1/2)}
	\begin{lstlisting}
let rec dijkstra_aux : float array ->
  (int -> (int * float) list) ->
  @r{int list FloatMap.t}@ -> unit = fun d e q ->
  match @r{FloatMap.min_binding q}@ with
  | exception Not_found -> ()
  | (w, us) -> dijkstra_aux d e @@
      List.fold_left (fun q u ->
        if d.(u) < w then q
        else List.fold_left (fun q (v, c) ->
          if d.(v) <= w +. c then q
          else begin
            d.(v) <- w +. c;
            @r{FloatMap.add (w +. c)
              (v :: try FloatMap.find (w +. c) q
                    with Not_found -> []) q}@
          end) q (e u)) (@r{FloatMap.remove w q}@) us
\end{lstlisting}
\end{frame}

\begin{frame}[fragile]
	\frametitle{Mapをヒープとして使う(2/2)}
	\begin{lstlisting}
let dijkstra : int ->
  (int -> (int * float) list) ->
  int -> int -> float = fun n e s ->
  let d = Array.make n infinity in
  d.(s) <- 0.;
  dijkstra_aux d e (@r{FloatMap.singleton 0. [s]}@);
  fun v -> d.(v)
\end{lstlisting}
	\begin{itemize}
		\item 標準ライブラリのMapは$O(\log N)$で\\最小値を取り出せる
			\begin{itemize}
				\item 計算量は$O(E \log V)$に
			\end{itemize}
	\end{itemize}
\end{frame}

\section{純粋関数型に}

\begin{frame}[fragile]
	\frametitle{最短距離もMapで管理してみる(1/2)}
	\begin{lstlisting}
let rec dijkstra_aux : (int ->(int*float)list)->
  @r{float IntMap.t}@ * int list FloatMap.t->
  @r{float IntMap.t}@ = fun e (d, q) ->
  match FloatMap.min_binding q with
  | exception Not_found -> d
  | (w, us) -> dijkstra_aux e @@
      List.fold_left (fun (d, q) u ->
        if @r{IntMap.find u d}@ < w then (d, q)
        else List.fold_left (fun (d, q) (v, c)->
          if @r{try IntMap.find v d <= w +. c
             with Not_found -> false}@ then (d, q)
          else @r{IntMap.add v (w +. c) d}@,
            FloatMap.add (w +. c)
              (v :: try FloatMap.find (w +. c) q
                    with Not_found -> []) q)
          (d,q) (e u))(d,FloatMap.remove w q) us
\end{lstlisting}
\end{frame}

\begin{frame}[fragile]
	\frametitle{最短距離もMapで管理してみる(2/2)}
	\begin{lstlisting}
let rec dijkstra :
  (int -> (int * float) list) ->
  int -> int -> float = fun e s ->
  let d = dijkstra_aux e (@r{IntMap.singleton s 0.}@,
     FloatMap.singleton 0. [s]) in
  fun v -> @r{try IntMap.find v d
           with Not_found -> infinity}@
\end{lstlisting}
	\begin{itemize}
		\item 破壊的代入が不要に
			\begin{itemize}
				\item 計算量は$O(E \log V)$のまま
			\end{itemize}
		\item 頂点数を与える必要がなくなった
		\item ファンクタを使えば辺の重みだけでなく,頂点の型も自由に選べる
	\end{itemize}
\end{frame}

\begin{frame}[fragile]
	\vspace{-7mm}
	\frametitle{実際に使ってみる(1/2)}
		\[\xymatrix{
			& v_{4} \ar@{-}[rd]^-{6} & & v_{6} \ar@{->}[d]^-{1} \\
			v_{5} \ar@{-}[ru]^-{9} \ar@{-}[r]^-{2} & v_{2} \ar@{-}[r]^-{11} & v_{3} & v_{7} \ar@{->}[d]^-{1} \\
			\alert{v_{0}} \ar@{-}[u]^-{14} \ar@{-}[r]_-{7} \ar@{-}[ru]^-{9} & v_{1} \ar@{-}[u]^-{10} \ar@{-}[ru]^-{15} & & \vdots \\
		} \]
	\vfill

	\begin{lstlisting}
# List.init 6 (dijkstra (function ... ) 0)
- : float list = [0.; 7.; 9.; 20.; 20.; 11.]
\end{lstlisting}
	\begin{center}
		\Large
		始点から到達可能な頂点が有限ならば,\\
		有限時間で最短距離を返す
	\end{center}
\end{frame}

\begin{frame}[fragile]
	\frametitle{実際に使ってみる(2/2)}
	\vspace{-7mm}
		\[\xymatrix{
			\vdots & \vdots & \vdots & \\
			v_{4} \ar@{-}[u]^-{3} \ar@{-}[r]^-{3} & v_{6} \ar@{-}[u]^-{4} \ar@{-}[r]^-{4} & v_{12} \ar@{-}[u]^-{5} \ar@{-}[r]^-{5} & \cdots \\
			v_{1} \ar@{-}[u]^-{2} \ar@{-}[r]^-{2} & v_{3} \ar@{-}[u]^-{3} \ar@{-}[r]^-{3} & v_{9} \ar@{-}[u]^-{4} \ar@{-}[r]^-{4} & \cdots \\
			v_{0} \ar@{-}[u]^-{1} \ar@{-}[r]^-{1} & v_{2} \ar@{-}[u]^-{2} \ar@{-}[r]^-{2} & v_{8} \ar@{-}[u]^-{3} \ar@{-}[r]^-{3} & \cdots \\
		} \]
	\vfill
	\begin{lstlisting}
# dijkstra (function ... ) 0 12
^CInterrupted.
\end{lstlisting}
	\begin{center}
		\Large
		始点から到達可能な頂点が無限だと無理
	\end{center}
\end{frame}

\section{無限グラフに使えるように}

\begin{frame}[fragile]
	\frametitle{改善策:計算の切り上げ}
  \begin{lstlisting}
let rec dijkstra : (int ->(int*float)list)->
  @r{int}@ -> float IntMap.t * int list FloatMap.t->
  @r{float}@ = fun e @r{t}@ (d, q) ->
  match FloatMap.min_binding q with
  | exception Not_found -> @r{infinity}@
  | (w, us) ->
      @r{if List.exists (( = ) t) us then w
      else}@ dijkstra e t @@
        List.fold_left (
          ...
        ) (d, FloatMap.remove w q) us

let dijkstra e s @r{t}@ = dijkstra e @r{t}@
  (IntMap.singleton s 0.,
   FloatMap.singleton 0. [s])
\end{lstlisting}
\end{frame}

\begin{frame}[fragile]
	\frametitle{使用例(1/2)}
	\vspace{-7mm}
		\[\xymatrix{
			\vdots & \vdots & \vdots & \\
			v_{4} \ar@{-}[u]^-{3} \ar@{-}[r]^-{3} & v_{6} \ar@{-}[u]^-{4} \ar@{-}[r]^-{4} & v_{12} \ar@{-}[u]^-{5} \ar@{-}[r]^-{5} & \cdots \\
			v_{1} \ar@{-}[u]^-{2} \ar@{-}[r]^-{2} & v_{3} \ar@{-}[u]^-{3} \ar@{-}[r]^-{3} & v_{9} \ar@{-}[u]^-{4} \ar@{-}[r]^-{4} & \cdots \\
			v_{0} \ar@{-}[u]^-{1} \ar@{-}[r]^-{1} & v_{2} \ar@{-}[u]^-{2} \ar@{-}[r]^-{2} & v_{8} \ar@{-}[u]^-{3} \ar@{-}[r]^-{3} & \cdots \\
		} \]
	\vfill
	\begin{lstlisting}
# dijkstra (function ... ) 0 12
- : int = 10
\end{lstlisting}
	\begin{center}
		\large
		終点までの距離が分かった時点で切り上げればよい
	\end{center}
\end{frame}

\begin{frame}[fragile]
	\frametitle{使用例(2/2)}
	\vspace{-7mm}
		\[\xymatrix{
			\vdots & \vdots & \vdots & \\
			v_{4} \ar@{-}[u]^-{3} \ar@{-}[r]^-{3} & v_{6} \ar@{-}[u]^-{4} \ar@{-}[r]^-{4} & v_{12} \ar@{-}[u]^-{5} \ar@{-}[r]^-{5} & \cdots \\
			v_{1} \ar@{-}[u]^-{2} \ar@{-}[r]^-{2} & v_{3} \ar@{-}[u]^-{3} \ar@{-}[r]^-{3} & v_{9} \ar@{-}[u]^-{4} \ar@{-}[r]^-{4} & \cdots \\
			v_{0} \ar@{-}[u]^-{1} \ar@{-}[r]^-{1} & v_{2} \ar@{-}[u]^-{2} \ar@{-}[r]^-{2} & v_{8} \ar@{-}[u]^-{3} \ar@{-}[r]^-{3} & \cdots \\
		} \]
	\vfill
	\begin{lstlisting}
# List.init 10000 (dijkstra (function ... ) 0)
^CInterrupted.
\end{lstlisting}
	\begin{center}
		\large
		計算の途中経過が捨てられるため,とても遅い
	\end{center}
\end{frame}

\section{OCamlは手続き型言語}

\begin{frame}[fragile]
	\frametitle{OCamlは手続き型言語(1/2)}
	\begin{itemize}
		\item 計算の途中経過を,クロージャの中に参照として覚えておく
	\end{itemize}
  \begin{lstlisting}
let dijkstra e s =
  @r{let d = ref @@ IntMap.singleton s 0. in
  let q = ref @@ FloatMap.singleton 0. [s] in}@
  let rec dijkstra_aux t =
    @r{let ans = try IntMap.find t !d
              with Not_found -> infinity in}@
    match FloatMap.min_binding !q with
    | exception Not_found -> ans
    | (w, us) ->
        @r{if ans <= w then ans
        else}@ begin
	\end{lstlisting}
\end{frame}
\begin{frame}[fragile]
	\frametitle{OCamlは手続き型言語(2/2)}
  \begin{lstlisting}
  q @r{:=}@ FloatMap.remove w !q;
  List.iter (fun u ->
    if w <= IntMap.find u !d then
      List.iter (fun (v, c) ->
        if
          try w +. c < IntMap.find v !d
          with Not_found -> true
        then begin
          d @r{:=}@ IntMap.add v (w +. c) !d;
          q @r{:=}@ FloatMap.add (w +. c)
            (v :: try FloatMap.find (w +. c) !q
                  with Not_found -> []) !q
        end) (e u)) us;
  dijkstra_aux t
end in dijkstra_aux
\end{lstlisting}
\end{frame}

\begin{frame}[fragile]
	\frametitle{解決!}
	\vspace{-7mm}
		\[\xymatrix{
			\vdots & \vdots & \vdots & \\
			v_{4} \ar@{-}[u]^-{3} \ar@{-}[r]^-{3} & v_{6} \ar@{-}[u]^-{4} \ar@{-}[r]^-{4} & v_{12} \ar@{-}[u]^-{5} \ar@{-}[r]^-{5} & \cdots \\
			v_{1} \ar@{-}[u]^-{2} \ar@{-}[r]^-{2} & v_{3} \ar@{-}[u]^-{3} \ar@{-}[r]^-{3} & v_{9} \ar@{-}[u]^-{4} \ar@{-}[r]^-{4} & \cdots \\
			v_{0} \ar@{-}[u]^-{1} \ar@{-}[r]^-{1} & v_{2} \ar@{-}[u]^-{2} \ar@{-}[r]^-{2} & v_{8} \ar@{-}[u]^-{3} \ar@{-}[r]^-{3} & \cdots \\
		} \]
	\vfill
	\begin{lstlisting}
# List.init 10000 (dijkstra (function ... ) 0)
- : float list =
[0.; 1.; 1.; 3.; 3.; 6.; 6.; 10.; 3.; 6.; ...]
\end{lstlisting}
	\begin{center}
		\large
		計算の途中経過が再利用され,高速になった
	\end{center}
\end{frame}

\section*{まとめ}

\begin{frame}
	\vspace{-7mm}
	\frametitle{まとめ}
	\begin{itemize}
		\item OCamlにより,様々な好ましい性質を持つダイクストラ法の実装を与えた
			\begin{itemize}
				\item 疎なグラフに対して高速
				\item ある始点からの最短距離を一度に計算
				\item 無限グラフにも対応可
			\end{itemize}
	\end{itemize}

	\vfill

	{\large 元ネタのソースコード:}
	\url{https://github.com/fetburner/compelib/blob/master/graph/dijkstra.ml}
	
	\vfill

	{\large AtCoderでの使用例:}
	\url{https://beta.atcoder.jp/contests/abc012/submissions/3198374}
\end{frame}

\end{document}
